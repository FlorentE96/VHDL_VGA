\documentclass{article}
\usepackage[francais]{babel}
\usepackage[utf8]{inputenc}
\usepackage[T1]{fontenc}
\usepackage{graphicx}
\usepackage{hyperref}
\usepackage{xcolor}
\usepackage{chngcntr}
\counterwithin*{section}{part}

\title{%
	Projet d'électronique numérique \\
	\large{visualiations musicales sur écran VGA}}
\date\today
\author{
	\bsc{LESTANDI Nathan}
	\and
	\bsc{ECOCHARD Florent}
}

\begin{document}
    \pagenumbering{gobble}
    \maketitle
    \newpage
    \tableofcontents
    \pagenumbering{arabic}
    \newpage
    \part*{Introduction}
    \addcontentsline{toc}{part}{Introduction}
    Dans le cadre de l'approffondissment de nos connaissance du langage de description VHDL au cours du semestre 7, nous avons du r
    \section{Cahier des charges}
    
    
    
    \section{Architecture}
    \begin{figure}[h]
    \includegraphics[width=\textwidth, keepaspectratio=true]{global.png}
    \caption{\label{global} Schéma global du système}
    \end{figure}
    Dans la figure~\ref{global} page~\pageref{global}, …
    \newpage
    \part{DSP : analyse du son}
    \section{Filtre passe-bas}
    \subsection{modélisation et dimension du filtre}
    Afin de réaliser un filtre passe-bas le moins encombrant possible et le plus facile à paramatré, nous avons utilisé un filtre exponentielle du première ordre qui s'écrit de la manière suivante: 
	\begin{center}
		\begin{equation}
			y(n)=(1-\epsilon)y(n-1)+x(n)\times\epsilon ,\epsilon \in [0,1]
		\end{equation}
	\end{center}    
    ce filtre correspond à un filtre RC,il ne possède qu'une seul variable : $\epsilon$.\\ Cependant sous cette forme un déphasage est induit dans les haute fréquences. Pour remédier à ce problème,nous allons moyenner l'entrée avec l'échantillon précédent soit :
	\begin{center}
		\begin{equation}
			y(n)=(1-\epsilon)y(n-1)+\frac{x(n)+x(n-1)\times\epsilon}{2}
		\end{equation}
	\end{center}  
    
    Sachant qu'on travail sur une architecture numérique, il faut éviter les multiplication/divisions. Pour déterminer epsilon nous avons choisit de prendre des valeurs multiple de deux afin de limiter les opérations à des décalages.
    
    Les epsilons pour chaque filtre seront déterminer via (https://fiiir.com/ faire la ref). Pour le passe-bas un epsilon de $2^{-4}=0.625$ permettra de couper convenablement les HF tout laissant les opération être de simple décalage
   	%image
    \subsection{simulation}
	Nous avons décidé de préciser la simulation du passe-bas car c'est le bloc de base des autres filtres et comprendre son fonctionnement est indispensable.
    
    Pour simuler ce bloc, nous avons rédiger un testbench basique où le signal en entrée est arbitraire, le reset et le signal enable sont aussi testés.
    
    %image
	
    \section{Filtre passe-haut}
    Pour le passe-haut nous allons utilisé 
    %image
    %\subsection{}
    %\paragraph{}
    \section{Filtre passe-bande}
    %image
    %\subsection{}
    %\paragraph{}
    \part{Visualisations, affichage}
    %\subsection{}
    %\paragraph{}    
    \section{L'"oscilloscope"}
    %\subsection{}
    %\paragraph{}    
    \section{Le "spectrogramme"}
    %\subsection{}
    %\paragraph{}    
    \section{Le bargraph}
    %\subsection{}
    %\paragraph{}
    \part*{Bilan}
    \addcontentsline{toc}{part}{Bilan}
  
\newpage
\bibliographystyle{plain} % Le style est mis entre accolades.
\bibliography{bibli} % mon fichier de base de données s'appelle bibli.bib
https://tomroelandts.com/articles/low-pass-single-pole-iir-filter
https://fiiir.com/
http://www.dspguide.com/ch19/2.htm
\addcontentsline{toc}{section}{Bibliographie}


\end{document}